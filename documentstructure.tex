\subsection{Document Structure}\label{subsec:documentstructure}
	After establishing some naming conventions in \S\ref{sec:conventions} we begin
by reviewing the required accuracy of COS FUV Target Acquisitions (TAs) in \S~\ref{sec:accuracy}.
In \S~\ref{sec:updated} we will review all of the TA parameters that need to be modified
for TA at LP3. This includes items in the flight software (FSW) and in the ground system (GS) commanding.
In \S~\ref{sec:spectrum_location}, we will review the actions required to locate the
FUV spectra at the desired location on the detector for lifetime position three (LP3).
In \S~\ref{sec:subarrays}, we detail the TA subarrays, and in \S~\ref{sec:focus} we discuss the
focus values to be used for COS FUV observations at LP3.

	We will then step through the observations of each of the five LENA3 (13636) visits used to determine
the TA parameters needed for operations at LP3. In \S~\ref{sec:visit1}, we discuss Visit~01, which
was designed to test the \texttt{ACQ/SEARCH} TA algorithm at LP3.
In \S~\ref{sec:visit2}, we discuss Visit~02, which was designed to test the \texttt{ACQ/PEAKD} TA procedure at LP3,
and also determines the WCA-to-SA offsets needed for the \texttt{ACQ/PEAKXD} procedure.
In \S~\ref{sec:visit3}, \S~\ref{sec:visit4}, and \S~\ref{sec:visit5}, we will review the
\texttt{ACQ/PEAKXD} procedure testing and plate scale determination for the G140L, G130M, and G160M
gratings, respectively. Visit~05 also contains the first on-orbit ``proof of concept''
\texttt{NUM\_POS} $>$ 1 \texttt{ACQ/PEAKXD} exposures. These will be discussed in \S~\ref{sec:new_peakxd}.
