\subsection{Introductory Notes and Conventions}
\vspace{-0.3cm}
There are a few COS conventions to be established before discussing TA in detail.
\begin{enumerate}
	\item{COS TAs are performed in raw or ``detector'' coordinates, not the ``user'' coordinate system of calibrated
		COS files. To avoid confusion over the different coordinate systems, we will use along-dispersion (AD) and cross-dispersion
		(XD) whenever possible.
		All references to the coordinates ``X'' and ``Y'' are in the detector coordinate system unless otherwise
		specified.
		In raw FUV coordinates, +X is -AD and +Y is +XD.
		The transformations between user and detector coordinates are :
		\begin{equation} FUV: X_{user} = 16383 - X_{detector} \ ; Y_{user} = Y_{detector} \end{equation}
		}
	\item{To clarify the names and locations of TA parameters, the following convention will be used:
		\begin{itemize}
			\item{COS TA modes and their Astronomers Proposal Tool (APT) optional parameters will be in \texttt{Courier} (e.g., \tacq{IMAGE} and \numpos).
			}
			\item{Keywords in FITS headers will be in \textit{ITALICIZED ALL CAPITALS} (e.g., \textit{ACQSLEWY}).
			}
			\item{Flight SoftWare (FSW) parameters and routines will be in \textsc{small capitals}.
			All TA FSW patchable constants begin with ``\textsc{pcta\_}'' (e.g., \textsc{pcta\_CalTargetOffset}).
			In this ISR, this prefix is considered implied after the initial introduction of a \textsc{pcta\_} paramater, and will not always be included.
			FSW patchable constants relating to mechanism positions begin with \textsc{pcmech\_} and will always be included in references.
			}
			\item{Archived COS files are in FITS (.fits) format. FITS filenames, or portions of a filename, will be in {\sf sans-serif} (e.g., {\sf ld9mg2nrq\_rawtag.fits} or {\sf \_spt.fits}).
			COS filenames are in the form {\sf IPPPSSOOT\_{\it extension}.fits}.
			The HST naming convention breaks down for COS as I=Instrument=``L'', PPP=Program ID, SS=Visit ID, OO=Exposure ID,
			and T=``Q'' for nominally recorded observations. See the COS Data Handbook (DHB, Fox et al. 2015) for a full breakdown of the HST IPPPSSOOT naming conventions.
			COS TA files have the {\it extension} of {\sf rawacq}, and additional
			information useful for TA analysis is contained in the {\sf IPPPSSOOT\_{\it spt}.fits} files known as the support file,
			and in the {\sf IPPPSSOOT\_{\it jit/f}.fits} files known as the jitter files.
			}
		\end{itemize}
	}
	%
	%		\begin{itemize}
	%			\item{Keywords in FITS headers will be in \texttt{ALL CAPITALS}.}
	%			\item{COS TA modes and optional parameters will be in \texttt{Courier}.}
	%			\item{Parameters in the COS flight software (FSW) will be in \textsc{small capitals}.
	%			In the FSW, all patchable constant TA parameters begin with ``\textsc{pcta\_}'', and all
	%			mechanical parameters begin with ``\textsc{pcmech\_}''. In this ISR, this prefix is considered implied and is not included
	%			after the first reference, except in the tables and descriptions in the appendices.}
	%			\item{COS FITS filenames, or portions of a filename, will be in {\sf sans-serif}.}
	%		\end{itemize}
	%	}
	\item{There are three centering options during \tacq{SEARCH} and \tacq{PEAKD}. In the Astronomers Proposal Tool (APT), these are
		referred to as \texttt{CENTER}=\texttt{FLUX-WT}, \texttt{FLUX-WT-FLR}, and \texttt{BRIGHTEST}.
		These parameters have slightly different names in the IHB, the FITS keywords, and the FSW.
		In this ISR, we will refer to the centering options as \texttt{CENTER}= \texttt{Flux-Weighted (FW)},
		\texttt{Flux-Weighted-Floor (FWF)}, and \texttt{Return-To-Brightest (RTB)}.
	}
	\item{When discussing the various subarrays used during COS TA, boxes will be specified by giving the lowest
		valued corner (C) and full size (S) for both X and Y. A box is fully specified by
		giving its XC, XS, YC, \& YS. In this ISR, these will always be given in detector coordinates.}
	\item{We use DE to reference the FUV digital elements (DE) in raw coordinates. There are no physical pixels on the FUV detector, and in raw coordinates the digital elements are of variable physical size. After geometric and thermal correction the digital elements correspond to a fixed physical size of approximately 6x24$\mu m$. In the pixelated user coordinates, we often refer to FUV elements as pixels (p). }
	\item{Milli-arcseconds (0.001\arcsec) will be abbreviated as mas. }
	\item{When referring to a particular day, we will use YEAR.DAY. For example, day 60 of 2010 will be referred to as 2010.060.
	We will also occasionally use decimal years. In these cases, there will only be a single digit in the fractional part (e.g., 2009.9).}
	\item{Archived COS files are in FITS format and the filenames are in the form {\sf IPPPSSOOT\_{\it extension}.fits}.
		The HST naming convention breaks down for COS as I=Instrument=``L'', PPP=Program ID, SS=Visit ID, OO=Exposure ID,
		and T=``Q'' for nominally recorded observations. See the COS IHB for a full breakdown of the HST IPPPSSOOT naming conventions.
		COS TA files have the {\it extension} of {\sf rawacq}, and additional TA information is contained in the
		{\sf IPPPSSOOT\_{\it spt}.fits} file known as the support file.}
\end{enumerate}
