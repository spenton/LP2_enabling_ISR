% RCS_ID: $Id$
\documentclass{stsci_report}
\usepackage{graphicx}
\usepackage{threeparttable}
\usepackage{ulem}
\usepackage[singlelinecheck=false]{caption}
%\usepackage{amsfonts}
\usepackage{latexsym}
%\usepackage{siunitx}
\usepackage{xcolor}
%\usepackage{tabularx}
\usepackage{multirow}
\usepackage{booktabs}
\usepackage{times}
\usepackage{deluxetable}
\usepackage{longtable}
\usepackage[colorlinks=true,linkcolor=blue]{hyperref}
\captionsetup[table]{singlelinecheck=false}
%\usepackage{fancyheadings}
%\pagestyle{fancy}

%\bibliography{bibliography.bib}
\copyrighttext{Copyright\copyright\ \the\year\ The Association of Universities for Research in Astronomy, Inc. All Rights Reserved.}


\newcommand{\tamonISR}[1]{COS ISR 2018-XXX}
\newcommand{\thisISR}[1]{COS ISR 2018-YYY}

\presubtitle{Instrument Science Report \thisISR}
\title{\textbf{FENA4: FUV Target Acquisition at the 2$^{\bf nd}$ Lifetime Position (LP2)}}
\author{Steven V. Penton}
\date{\today}

\DeclareGraphicsRule{.eps}{eps}{.eps}{}
\DeclareGraphicsRule{.ps}{eps}{.ps}{}

\definecolor{green}{rgb}{0, 1.0, 0}
\definecolor{red}{rgb}{1,0,0}
\definecolor{magenta}{rgb}{1,0,1}
\definecolor{blue}{rgb}{0,0,1}
\definecolor{lblue}{rgb}{0.8,0.85,1}
\definecolor{darkgreen}{rgb}{0.25,1.0,0.25}
\definecolor{brown}{cmyk}{0, 0.8, 1, 0.6}
\definecolor{yellow}{rgb}{1, 1, 0}
\definecolor{light}{gray}{.80}
\definecolor{dark}{gray}{.20}

%\setlength{\headheight}{5mm}
%\setlength{\headsep}{10mm}
%\setlength{\footskip}{10mm}
%\setlength{\textheight}{220mm}
%\setlength{\textwidth}{170mm}
%\setlength{\topmargin}{-8.0mm}
%\setlength{\oddsidemargin}{+6.0mm}
%\setlength{\evensidemargin}{+6.0mm}
%\setlength{\parskip}{1mm}
%\setlength{\parsep}{100mm}
%\setlength{\parindent}{10mm}

\newenvironment{mylisting}
{\begin{list}{}{\setlength{\leftmargin}{1em}}\item\scriptsize\bfseries}
{\end{list}}

\newenvironment{mytinylisting}
{\begin{list}{}{\setlength{\leftmargin}{1em}}\item\tiny\bfseries}
{\end{list}}

%\def\ssection#1{\section{\hbox to \hsize{\large\bf #1\hfill}}}
%\def\ssectionstar#1{\section*{\hbox to \hsize{\large\bf #1\hfill}}}
%\def\ssubsection#1{\subsection{\hbox to \hsize{\normalsize\bfseries\itshape #1\hfill}}}
%\def\ssubsectionstar#1{\subsection*{\hbox to \hsize{\normalsize\bfseries\itshape #1\hfill}}}
%\def\ssubsubsection#1{\subsection{\hbox to \hsize{\normalsize\it #1\hfill}}}
%\def\ssubsubsectionstar#1{\subsection*{\hbox to \hsize{\normalsize\it #1\hfill}}}

% define some dates
\newcommand{\firstdate}{2009.215}
\newcommand{\lastdate}{2010.059}
\newcommand{\lptwosiafdate}{2010.060}
\newcommand{\lpthreesiafdate}{2014.060}
\newcommand{\reallastdate}{2010.119}
\newcommand{\finallastdate}{2010.285}
\newcommand{\lptwopatchdate}{2012.XXX}
\newcommand{\lpthreepatchdate}{2014.300}

% define some useful shortcuts

\def\arcsec{\hbox{$^{\prime\prime}$}}
\def\degree{\hbox{$^{\circ}$}}
\newcommand{\nokmsno}{{\rm km~s}\ensuremath{^{-1}}}
\newcommand{\kmsno}{~\nokmsno}
\newcommand{\kms}{~\nokmsno\ }
\newcommand{\numpos}{\texttt{NUM$\_$POS}}
\newcommand{\numposone}{\texttt{NUM$\_$POS=1}}
\newcommand{\numposgtone}{\texttt{NUM$\_$POS$>$1}}
\newcommand{\tacq}[1]{\texttt{ACQ/#1}}
\newcommand{\acq}[1]{\texttt{ACQ/#1}}
\newcommand{\pid}[1]{{\rm P}#1}
\newcommand{\plamp}[1]{{\bf P#1}}
\newcommand{\pr}[1]{{\it PR\#}#1}
\newcommand{\fsw}[1]{{\textsc{LTA#1}}}
\newcommand{\cenwave}{\textit{CENWAVE}}
\newcommand{\ts}{\textsuperscript}
\def\cenwaves{\textit{CENWAVE}s}

% LP2 specific section
\newcommand{\blindad}{-0.XX$\pm$0.YY}
\newcommand{\blindxd}{-0.18$\pm$0.68}

% Define the new \sqrt in terms of the old one
\let\oldsqrt\sqrt
\def\sqrt{\mathpalette\DHLhksqrt}
\def\DHLhksqrt#1#2{%
\setbox0=\hbox{$#1\oldsqrt{#2\,}$}\dimen0=\ht0
\advance\dimen0-0.2\ht0
\setbox2=\hbox{\vrule height\ht0 depth -\dimen0}%
{\box0\lower0.4pt\box2}}

\begin{document}
\maketitle
COS is a slit-less spectrograph with a pair of circular apertures with radii of 1.25\arcsec. The Primary Science Aperture (PSA) is
placed at the location of the minimum beam waist, with the instrument focused on the sky. The Bright Object Aperture (BOA) is
identical to the PSA except that the aperture is filled with an $\sim$ND2 filter.
To achieve our FUV centering goals, we need to center a point source to within 0.3\arcsec\ in the cross-dispersion (XD) direction and 0.106\arcsec\ in the along-dispersion (AD) direction.

In this ISR we will discuss the performance of the COS FUV spectroscopic on-board target acquisition (TA) modes at the second lifetime position (LP2). Except for the obvious displacement of the TA sub-arrays is the Y direction, the only upadates required are to the grating specific plate scales and the WCA-to-PSA/BOA offsets.

\clearpage
\tableofcontents
\listoffigures
\listoftables
\newpage
\section{Introduction}

This ISR details the execution of HST COS Program 12797 - Second COS FUV Lifetime Position: FUV Target Acquisition Parameter Update (FENA4). This ISR is designed to supplement information about COS FUV spectroscopic Target Acqusitions (TAs) contained in the Instrument Handbook (IHB(v5), Holland 2012) , COS ISR 2010-014(v1) (Keyes \& Penton, ``COS Target Acquisition Guidelines, Recommendations, and Interpretation'', 2010), and COS TIR 2010-03(v1) (Penton \& Keyes, ``On-Orbit Target Acquisitions with HST+COS'', 2010).

FENA4, was the last of a series of programs allowing COS to be moved to its second lifetime position (LP2). These other programs, and what they contributed to this program, were;
\begin{itemize}
\item{12793} - FUV Detector High Voltage Sweep (FENA1)
	\begin{itemize}
	\item{Determines the initial FUV Detector High Voltage Setting for LP2.}
	\end{itemize}
\item{12795} - Verification of Aperture and FUV Spectrum Placement (FENA2)
	\begin{itemize}
	\item{Provides the cross-dispersion (XD) spectral locations for each grating.}
	\item{Provides the initial estimate for the WCA-to-PSA offsets for each grating.}
	\item{Provides the inital estimate for the XD plate scales.}
	\end{itemize}
\item{12796} - Focus Sweep Enabling Program (FENA3)
	\begin{itemize}
	\item{Determines the FUV focus settings for use during FUV spectroscopic TA.}
	\end{itemize}
\end{itemize}

bout performing FUV Spectroscopic Target Acquisitions (TAs) with the Cosmic Origins Spectrograph (COS).

\subsection{Document Structure}\label{subsec:documentstructure}
	After establishing some naming conventions in \S\ref{sec:conventions} we begin
by reviewing the required accuracy of COS FUV Target Acquisitions (TAs) in \S~\ref{sec:accuracy}.
In \S~\ref{sec:updated} we will review all of the TA parameters that need to be modified
for TA at LP3. This includes items in the flight software (FSW) and in the ground system (GS) commanding.
In \S~\ref{sec:spectrum_location}, we will review the actions required to locate the
FUV spectra at the desired location on the detector for lifetime position three (LP3).
In \S~\ref{sec:subarrays}, we detail the TA subarrays, and in \S~\ref{sec:focus} we discuss the
focus values to be used for COS FUV observations at LP3.

	We will then step through the observations of each of the five LENA3 (13636) visits used to determine
the TA parameters needed for operations at LP3. In \S~\ref{sec:visit1}, we discuss Visit~01, which
was designed to test the \texttt{ACQ/SEARCH} TA algorithm at LP3.
In \S~\ref{sec:visit2}, we discuss Visit~02, which was designed to test the \texttt{ACQ/PEAKD} TA procedure at LP3,
and also determines the WCA-to-SA offsets needed for the \texttt{ACQ/PEAKXD} procedure.
In \S~\ref{sec:visit3}, \S~\ref{sec:visit4}, and \S~\ref{sec:visit5}, we will review the
\texttt{ACQ/PEAKXD} procedure testing and plate scale determination for the G140L, G130M, and G160M
gratings, respectively. Visit~05 also contains the first on-orbit ``proof of concept''
\texttt{NUM\_POS} $>$ 1 \texttt{ACQ/PEAKXD} exposures. These will be discussed in \S~\ref{sec:new_peakxd}.

\clearpage
%\subsection{Introductory Notes and Conventions}
\vspace{-0.3cm}
There are a few COS conventions to be established before discussing TA in detail.
\begin{enumerate}
	\item{COS TAs are performed in raw or ``detector'' coordinates, not the ``user'' coordinate system of calibrated
		COS files. To avoid confusion over the different coordinate systems, we will use along-dispersion (AD) and cross-dispersion
		(XD) whenever possible.
		All references to the coordinates ``X'' and ``Y'' are in the detector coordinate system unless otherwise
		specified.
		In raw FUV coordinates, +X is -AD and +Y is +XD.
		The transformations between user and detector coordinates are :
		\begin{equation} FUV: X_{user} = 16383 - X_{detector} \ ; Y_{user} = Y_{detector} \end{equation}
		}
	\item{To clarify the names and locations of TA parameters, the following convention will be used:
		\begin{itemize}
			\item{COS TA modes and their Astronomers Proposal Tool (APT) optional parameters will be in \texttt{Courier} (e.g., \tacq{IMAGE} and \numpos).
			}
			\item{Keywords in FITS headers will be in \textit{ITALICIZED ALL CAPITALS} (e.g., \textit{ACQSLEWY}).
			}
			\item{Flight SoftWare (FSW) parameters and routines will be in \textsc{small capitals}.
			All TA FSW patchable constants begin with ``\textsc{pcta\_}'' (e.g., \textsc{pcta\_CalTargetOffset}).
			In this ISR, this prefix is considered implied after the initial introduction of a \textsc{pcta\_} paramater, and will not always be included.
			FSW patchable constants relating to mechanism positions begin with \textsc{pcmech\_} and will always be included in references.
			}
			\item{Archived COS files are in FITS (.fits) format. FITS filenames, or portions of a filename, will be in {\sf sans-serif} (e.g., {\sf ld9mg2nrq\_rawtag.fits} or {\sf \_spt.fits}).
			COS filenames are in the form {\sf IPPPSSOOT\_{\it extension}.fits}.
			The HST naming convention breaks down for COS as I=Instrument=``L'', PPP=Program ID, SS=Visit ID, OO=Exposure ID,
			and T=``Q'' for nominally recorded observations. See the COS Data Handbook (DHB, Fox et al. 2015) for a full breakdown of the HST IPPPSSOOT naming conventions.
			COS TA files have the {\it extension} of {\sf rawacq}, and additional
			information useful for TA analysis is contained in the {\sf IPPPSSOOT\_{\it spt}.fits} files known as the support file,
			and in the {\sf IPPPSSOOT\_{\it jit/f}.fits} files known as the jitter files.
			}
		\end{itemize}
	}
	%
	%		\begin{itemize}
	%			\item{Keywords in FITS headers will be in \texttt{ALL CAPITALS}.}
	%			\item{COS TA modes and optional parameters will be in \texttt{Courier}.}
	%			\item{Parameters in the COS flight software (FSW) will be in \textsc{small capitals}.
	%			In the FSW, all patchable constant TA parameters begin with ``\textsc{pcta\_}'', and all
	%			mechanical parameters begin with ``\textsc{pcmech\_}''. In this ISR, this prefix is considered implied and is not included
	%			after the first reference, except in the tables and descriptions in the appendices.}
	%			\item{COS FITS filenames, or portions of a filename, will be in {\sf sans-serif}.}
	%		\end{itemize}
	%	}
	\item{There are three centering options during \tacq{SEARCH} and \tacq{PEAKD}. In the Astronomers Proposal Tool (APT), these are
		referred to as \texttt{CENTER}=\texttt{FLUX-WT}, \texttt{FLUX-WT-FLR}, and \texttt{BRIGHTEST}.
		These parameters have slightly different names in the IHB, the FITS keywords, and the FSW.
		In this ISR, we will refer to the centering options as \texttt{CENTER}= \texttt{Flux-Weighted (FW)},
		\texttt{Flux-Weighted-Floor (FWF)}, and \texttt{Return-To-Brightest (RTB)}.
	}
	\item{When discussing the various subarrays used during COS TA, boxes will be specified by giving the lowest
		valued corner (C) and full size (S) for both X and Y. A box is fully specified by
		giving its XC, XS, YC, \& YS. In this ISR, these will always be given in detector coordinates.}
	\item{We use DE to reference the FUV digital elements (DE) in raw coordinates. There are no physical pixels on the FUV detector, and in raw coordinates the digital elements are of variable physical size. After geometric and thermal correction the digital elements correspond to a fixed physical size of approximately 6x24$\mu m$. In the pixelated user coordinates, we often refer to FUV elements as pixels (p). }
	\item{Milli-arcseconds (0.001\arcsec) will be abbreviated as mas. }
	\item{When referring to a particular day, we will use YEAR.DAY. For example, day 60 of 2010 will be referred to as 2010.060.
	We will also occasionally use decimal years. In these cases, there will only be a single digit in the fractional part (e.g., 2009.9).}
	\item{Archived COS files are in FITS format and the filenames are in the form {\sf IPPPSSOOT\_{\it extension}.fits}.
		The HST naming convention breaks down for COS as I=Instrument=``L'', PPP=Program ID, SS=Visit ID, OO=Exposure ID,
		and T=``Q'' for nominally recorded observations. See the COS IHB for a full breakdown of the HST IPPPSSOOT naming conventions.
		COS TA files have the {\it extension} of {\sf rawacq}, and additional TA information is contained in the
		{\sf IPPPSSOOT\_{\it spt}.fits} file known as the support file.}
\end{enumerate}

\vspace{-0.4cm}
\section{Required Accuracy of COS FUV TA}\label{sec:accuracy}
\vspace{-0.3cm}
The original Contract End Item (CEI) requirement (STE-63) for COS TA is \\

\footnotesize
\noindent \texttt{During the target acquisition process, it shall be possible to calibrate the location of the science aperture in detector pixel
coordinates in order to compute a maneuver of HST needed to improve the centering of the target within the aperture.
The measurement shall be made using COS internal sources.
The measurement calibration process shall allow the target to be centered
in the science aperture with an accuracy of 0.3\arcsec.}\\

\normalsize
There is the additional FUV requirement that \\

\footnotesize
\noindent \texttt{Wavelengths assigned to data points in the fully reduced and calibrated COS spectra shall have an accuracy equivalent to an
absolute uncertainty of less than $\pm$15\kms in the $R=20,000$ modes, $\pm$ 150\kms in mode G140L}\\

\normalsize
For COS TA, we take the CEI spec (0.3\arcsec) to refer to the required centering accuracy in the XD, and the wavelength requirements to apply to
the TA accuracy required in the AD direction. Since the AD requirement is in units of \kmsno, it is detector and wavelength
dependent as defined in equations~\ref{eq:TAcenter}--\ref{eq:TAcenterL}.
Assuming that the wavelength error budget is split evenly between the COS TA and wavelength scale accuracies,
the error budgets for the COS gratings, in arc-seconds (\arcsec), are given in Table~\ref{table:TAaccuracy}. By ``evenly'' we mean that when added in quadrature the total error budget is that given by the second column of Table~\ref{table:TAaccuracy}. Setting the TA error budget equal to the wavelength scale accuracy, the AD TA requirement given in the third column is the second column divided by $\sqrt{2}$.
\small
\begin{eqnarray}\label{eq:TAcenter}
\Delta\ AD(G130M@1300\AA) = {{ 15\kms \times 1300\AA}\over{c \times 0.00997\AA/ p\times 43.5 p/\arcsec}} = 0.150\arcsec\\
\Delta\ AD(G160M@1600\AA) = {{ 15\kms \times 1600\AA}\over{c \times 0.01223\AA/ p\times 42.9 p/\arcsec}} = 0.153\arcsec\\
\Delta\ AD(G140L@1800\AA) = {{150\kms \times 1800\AA}\over{c \times 0.08030\AA/ p\times 45.4 p/\arcsec}} = 0.247\arcsec\label{eq:TAcenterL}
\end{eqnarray}
\normalsize
\clearpage

\begin{deluxetable}{|c||c|c|}
\tablewidth{0pt}
\tabcolsep 17pt
\tablecolumns{3}
\tablecaption{COS Along-Dispersion (AD) Centering Requirements.\label{table:TAaccuracy}}
\tablehead{	\colhead{COS Grating} &
			\colhead{Total Error Budget} &
			\colhead{AD TA Requirement}
}
\startdata
G130M & 0.150\arcsec\ & 0.106\arcsec\\
G160M & 0.153\arcsec\ & 0.108\arcsec\\
G140L & 0.247\arcsec\ & 0.175\arcsec\\
\enddata
\tablecomments{Assuming the total error budget (column 2) is split equally between TA AD centering and wavelength scale accuracy,
the AD TA requirements (column 3) are 1/$\sqrt{2}$ of the total error budget (equations~\ref{eq:TAcenter}--\ref{eq:TAcenterL}).}
\end{deluxetable}

\section{Initial ``Blind'' Pointing of HST}\label{sec:IP}
\vspace{-0.3cm}
We can estimate the initial (blind) pointing accuracy of HST+COS
observations by reverse engineering all of the TAs of the SMOV plus early
GTO and GO observations. Figure~\ref{TAblindpointing} shows the observed blind-pointing results for the three COS TA modes: NUV imaging, NUV spectroscopic, and FUV spectroscopic.
Overall, we observe an initial pointing bias of [AD,XD] = [\blindad,\blindxd]\arcsec.
In general, the $1\sigma$ accuracy of guide stars in the GSC2 is $\sim$0.4\arcsec\,
so the standard deviation of our blind pointing is slighter higher than the guide star accuracy due to target coordinate uncertainties.
The initial pointing bias of [-0.23,-0.18]\arcsec\ was, however, not expected and was the result of a minor FGS-to-COS misalignment (see \tamonISR).
%or a spacecraft or COS mechanism drift problem (the target is always drifting in the same direction during TA).

\clearpage
\section{COS Target Acquisition Modes}

In the next four sections, we will examine each of these modes in detail,
particularly when the TA centering accuracy depends upon the proper choice
of TA mode option. A summary of the recommended values of all TA mode options is given in \S~\ref{sec:Summary} (Table~\ref{table:TAoptions}). Each TA modes uses TA flight software (FSW) parameters which are not specifiable by the user. A description of each of these parameters is given in Appendix~\ref{sec:FSWdescription},
and the current and past values of the important TA FSW parameters are given in Appendix~\ref{sec:FSWparameters}.
\subsection{Using \tacq{SEARCH}}\label{sec:SEARCH}
An \tacq{SEARCH} is fully defined by specifying the three TA parameters
\texttt{SCAN-SIZE}, \texttt{STEP-SIZE}, and the \texttt{CENTER} method and
can be performed in imaging or spectroscopic made.
\texttt{SCAN-SIZE} can be 2, 3, 4, or 5 (the upper limit is set by the
TA parameter \textsc{pcta\_MaxScanSize}). \texttt{STEP-SIZE} can range from 0.001\arcsec\ to 2.0\arcsec\
(the FSW parameters \textsc{pcta\_MinStepSize} and \textsc{pcta\_MaxStepSize}).
The recommended value of \texttt{STEP-SIZE} is the default (1.767\arcsec).
This is the largest \texttt{STEP-SIZE} that covers the sky without internal gaps or holes in on-sky coverage,
and should be used in almost all cases. The \texttt{CENTER} method can be \texttt{FWF} (Flux-Weighted-Floor),
\texttt{FW} (Flux-Weighted), or \texttt{RTB} (Return to Brightest). \texttt{CENTER=FWF} is recommended for
\texttt{SCAN-SIZE}=3 or larger, while \texttt{CENTER=FW} is recommended for \texttt{SCAN-SIZE}=2.
Use of \texttt{CENTER=FWF} with \texttt{SCAN-SIZE}=2 produces an asymmetric pattern that pulls the
target away from proper centering.
\texttt{CENTER=FWF} centers better than \texttt{CENTER=FW} because it subtracts the lowest counts in any
dwell point from all counts, which acts as a crude background removal. A simple schematic of \texttt{STEP-SIZE}=2--5,
\texttt{SCAN-SIZE}=1.767\arcsec\ \tacq{SEARCH}s are given in Figure~\ref{Patterns}. The {\color{red}red} circle here has a
radius of 3\arcsec, and is useful in comparing the area on the sky covered by the available patterns.
{\color{blue}Blue} numbers show the order of the dwells in the pattern, and the pre-launch PSF is shown in the center for comparison.

Ray-tracing and computer simulations for the NUV (COS-11-0024A) and
FUV (COS-11-0016A) predict that properly designed \tacq{SEARCH}s should
center the target to within 0.1--0.2\arcsec\ in all modes in both AD and XD,
if the target was within the box on the sky contained by the outer dwell points.
These simulations were based upon the predicted, not the observed PSF, (see Figure~\ref{PSF} and COS ISR 2010-01).
The observed PSF is noticeably asymmetric and contains a much larger percentage
of the light in the extended wings. The PSF asymmetries
and extended wings, along with the extended transmission function of the aperture
(Figure~\ref{Transmission}), tend to feed incorrect information into the \tacq{SEARCH} centering
algorithm.
For these reasons, a single \tacq{SEARCH} should be not expected to center the target
to better than 0.3\arcsec\ in either AD and XD. Depending on where the target
falls in the \tacq{SEARCH} pattern, it could be as much as 0.5\arcsec\ off-center. This is why we recommend that
all TAs follow up their initial \tacq{SEARCH} with either an \tacq{IMAGE}, \tacq{PEAKXD}+\texttt{PEAKD}, or a second 2x2 \tacq{SEARCH}.

The NUV detector background has risen from about 60 counts/s to 400 counts/s since May 2009, with an annual rate of increase of $\sim$226 counts/year (see \S~\ref{sec:TAback}).
The original NUV imaging and spectroscopic TA subarrays comprised $\sim$13\% of the NUV detector and are were counting 31--46 counts/s of detector background
as of mid-2010.
Under the simplifying assumption that during an \tacq{SEARCH} the weakest background counts will be 1$\sigma$ low and the brightest will be 1$\sigma$ high, when using \texttt{CENTER=FWF} the target should be brighter than $2 \times \sqrt(\textrm{maximum background rate})$ to
avoid significant pointing error due to background contamination. For normally distributed background events and observations away from the SAA, this is $\sim$14 counts/s, or \tacq{SEARCH} dwells times of greater than $\sim$120s for target equal to S/N=40 TAs.
For cases using \texttt{CENTER=FW}, the target should be brighter than or equal to the maximum expected background rate to provide adequate centering (count rates greater than 50 counts/s and dwell times less than 35s).
In cases where the target does not meet these criteria, other COS TA strategies should be employed.\\

As discussed in \S~\ref{sec:TAback}, NUV imaging \tacq{SEARCH} TA subarrays were optimized (reduced) during mid-2010, and the
above warning about NUV \tacq{SEARCH}s now only apply to spectroscopic TAs.

\clearpage
\vspace{-0.3cm}
\subsection{\tacq{IMAGE}}\label{sec:acqimage}
\vspace{-0.3cm}
When mirrors are selected on both Optics Select Mechanisms (OSMs), COS can be used in NUV imaging mode. \tacq{IMAGE}
uses this mode to center a point-source, or the brightest point in an extended object, in the aperture.
All COS apertures are on a common aperture plate, and therefore have a fixed physical offset.
Flashing the on-board Pt-Ne wavelength calibration lamp allows the determination of the position
of the wavelength calibration aperture (WCA) on the NUV detector.
COS has two Pt-Ne lamps; \plamp{2} is used exclusively for TA and \plamp{1} is used for wavelength calibration. Once the position
of the WCA is known, the center of the other apertures can be precisely calculated.
NUV \tacq{IMAGE} observations can use either the PSA or BOA aperture and either MIRRORA or MIRRORB modes.
MIRRORA mode is simply using a flat mirror on OSM2.
This mirror has 2mm thick UV grade fused silica `order sorter' on the front which suppresses light at $\lambda < 1600 \AA$.

The order-sorter was placed in front of the gratings, so that the light makes two passes through the filter before reaching the detector.
UV grade fused silica transmits well for $\lambda > 1700\AA$, but is almost totally opaque for $\lambda$ < 1600$\AA$.
while MIRRORB is a slight rotation of OSM2 so that light reflects off the order-sorting filter in front of the mirror.
The TA order sorter has a slight wedge of $\sim$ 30\arcsec in the AD.
This produces two images, one from the front surface reflection off of the order sorter,
and a second reflecting off of the back surface of the order sorter.
The second reflection is dimmer by a factor of $\sim$ 0.48\footnote{Since the primary reflection does not pass through the order sorter, but the secondary image dose, this ratio is target SED dependent.}, therefore MIRRORB produces two target images in a $\sim$ 2:1 intensity ratio.
Each of these modes has different WCA-to-aperture offsets that are stored in the COS FSW
as the parameter \textsc{pcta\_CalTargetOffset} (Table~\ref{table:TAcaltargetoffset}).
Details of this TA step, known as \fsw{IMCAL}, are given in \S~\ref{sec:LTAIMCAL}.

\begin{figure}[htb]
\includegraphics[angle=90,scale=0.9,width=0.99\textwidth]{png/ordersorter.png}
\caption{Overview of the COS TA MIRROR hardware optics.\label{fig:ordersorter}}
\end{figure}

Once the desired position of the aperture+MIRROR center has been determined, an image of the sky is
taken to determine the initial target position. Using the plate scale parameters stored in the FSW
constants \textsc{pcta\_NUVMilliArcsecsPerPixelX} and \textsc{pcta\_NUVMilliArcsecsPerPixelY} (Table~\ref{table:TAscalar}),
the calculated target motion required to
center the target in pixels is converted to arc-seconds (\arcsec) and HST is commanded to slew the
required amount to center the target in the aperture. Details of this TA step, known
as \texttt{LTAIMAGE} are given in \S~\ref{sec:LTAIMAGE}.

After slewing to the desired location, \tacq{IMAGE} takes a second image of the sky
to verify that the target was centered as desired. The two images, and the results of the
WCA target location, are returned in the \texttt{IPPPSSOOT\_rawacq.fits} file. In \S~\ref{sec:IMAGEcentering},
we evaluate the performance of the four \texttt{APERTURE+MIRROR} \tacq{IMAGE}
modes and compare them to our centering requirements.
\vspace{-0.3cm}

\begin{deluxetable}{|r|r|r|r|r|}
\tablewidth{0pt}
\tabcolsep 6pt
\tablecolumns{5}
\tabletypesize{\footnotesize}
\tablecaption{On-Orbit \texttt{LTIMCAL} Measurements\label{table:tableWCA}.}
\tablehead{ \colhead{Aperture} & \colhead{MIRROR} & \colhead{AD Position}
& \colhead{XD Position} & \colhead{\# of \texttt{LTAIMCALs}}
}
\startdata
PSA & MIRRORA & 474.0 $\pm$ 15.5 &370.0 $\pm$ 0.6 &80 \\
PSA & MIRRORB & 689.0 $\pm$ 11.9 &209.0 $\pm$ 0.8 & 116 \\
\hline
BOA & MIRRORA & 474.0 $\pm$ 15.0 &370.0 $\pm$ 0.8 & 46 \\
BOA & MIRRORB & 684.0 $\pm$ 8.8 &208.0 $\pm$ 0.8 & 6 \\
\hline
ALL & MIRRORA & 474.0 $\pm$ 15.3 &370.0 $\pm$ 0.6 & 126 \\
ALL & MIRRORB & 689.0 $\pm$ 11.8 &209.0 $\pm$ 0.8 & 122 \\
\hline
\enddata
%Updated May 25, 2018
\tablecomments{The reported positions are the median detector coordinates.
The proposed subarrays shown in Figure~\ref{LTAIMCALpos} are derived from these measurements.}
\end{deluxetable}

The average number of counts in the WCA \texttt{LTAIMCAL} MIRRORA images was 3233$\pm$68
(a 7-second exposure with \plamp{2} at low current).
The average number of counts in the WCA \texttt{LTAIMCAL} MIRRORB images was 798$\pm$153
(a 30-second exposure with \plamp{2} at low current).
The average number of counts in the primary MIRRORB image was 2/3 of this value ($\sim535$).
As discussed in \S~\ref{sec:TAback}, the COS dark rate is currently $\sim$220--330 counts/s over the entire detector, or $\sim$31--46 counts/s in the WCA subarray.
The WCA MIRRORB exposures are currently 30 seconds in duration, so the total background in the subarray is $\sim930$ counts, comparable to the counts in the lamp images.

Figure~\ref{LTAIMCALrates} shows the \texttt{LTAIMCAL} counts as a function of time.
The solid crosses indicate the total number of accumulated counts during the {\color{green}MIRRORA}
and {\color{blue}MIRRORB} \texttt{LTAIMCAL}. The solid lines give the moving average of the measured counts.
There is significant scatter in the total \texttt{LTAIMCAL} counts due to detector background which
is related to the temperature of the NUV MAMA tube (\texttt{LNTUBET}). As discussed in
Pascucci et al. 2010, this relationship has been estimated as :
 \begin{equation}\label{eq:NUVcps}
 	{\rm NUV~Dark~Rate =}~ 1.1\times~10^9~~exp({{-8791}\over{T_c + 273.16}})~({{(t-t_0)}\over{163}} + 1.5)
 \end{equation}
where t$_0$ is the modified Julian date of 55100.0 (2009.269) and T$_c$ is the MAMA tube temperature tracked by the
COS keyword \texttt{LNTUBET}.
\begin{figure}[b]
\center
\includegraphics[angle=90,scale=1.0,width=0.99\textwidth]{eps/wca_positions_nota_ALL.eps}
\caption[WCA \texttt{LTAIMCAL} Positions]{\small Representation of the NUV detector showing the locations of the wavelength
calibration lamp images (WCA, \texttt{LTAIMCAL}) and the initial positions of the target images (PSA/BOA, \texttt{LTAIMAGE}).
The two solid boxes on the left show the current MIRRORA and MIRRORB WCA TA subarrays capturing the WCA lamp locations for both {\color{magenta}PSA} and {\color{blue}BOA} \tacq{IMAGE}S. The two solid boxes on the right show the
current MIRRORA and MIRRORB TA subarrays for PSA/BOA \texttt{LTAIMAGE}s. The initial target pointings are also shown
for the {\color{magenta}PSA} and {\color{blue}BOA} \tacq{IMAGE}s. The dashed boxes are proposed updates to these subarrays as discussed in \S~\ref{sec:TAback} and shown in Table~\ref{table:TAnuvIMAGEupdate}. All coordinates are in detector coordinates.\label{LTAIMCALpos}}
\end{figure}

\begin{figure}[b]
\center
\includegraphics[angle=90,scale=0.9,width=0.99\textwidth]{png/WCA_countrates.png}
\caption[WCA \texttt{LTAIMCAL} Counts]{\small Count rates during the \texttt{LTAIMCAL} TA phase for {\color{green}MIRRORA}
and {\color{blue}MIRRORB} observations (crosses). The solid lines are a moving average of the calibration lamp plus the detector background counts.
As explained in the text, from the date and the temperature of the NUV MAMA we can estimate the detector background present
in the \texttt{LTAIMCAL} subarrays. After subtracting this estimate of the detector background,
we determine the number of \texttt{LTAIMCAL} calibration lamp events as a function of date.
The wavelength calibration lamp used during LTIMCAL appears to be stable over the sampled interval.
The {\color{green}MIRRORA} count rates are clearly adequate, but the {\color{blue}MIRRORB} count rates
in the two {\color{blue}MIRRORB} images appear to be dropping and this exposure time should be monitored and adjusted for Cycle~19.\label{LTAIMCALrates}}
\end{figure}

\vspace{-0.3cm}
\subsection{\tacq{PEAKXD}}
\tacq{PEAKXD} sequences start with a 17-second flash of \plamp{2} at medium current to locate the wavelength calibration spectrum (WCS, or WCA spectrum) in dispersed light on the detector.
For NUV TAs, one must specify the stripe to be used in centering.
The optional parameter is {\it STRIPE=A, B} (DEFAULT), or {\it C}, or in APT, {\it STRIPE=SHORT, MEDIUM, LONG}.
For FUV TAs, one must specify which segment(s) to use ({\it SEGMENT=A} (DEFAULT for G140L){\it, B} or {\it BOTH} (DEFAULT for G130M, G160M).
To limit the amount of detector background contributing to the WCS location, a single subarray is used for the NUV detector and one for each of the FUV segments (if applicable).
The values of these WCS \tacq{PEAKXD} TA subarrays are given for the NUV and FUV in Tables~\ref{table:TAnuvPEAKXDxc} and \ref{table:TAsubWCAfuv} for the FUV channel.

Once the XD location of the NUV WCS for the {\it STRIPE} selected is determined (using a
median, \textsc{pcta\_UseMedian4CAL4NUV=TRUE}), the expected location of the median (\textsc{pcta\_UseMedian4PKXD4NUV=TRUE})
of the spectral stripe (BOA or PSA) is calculated using the grating and aperture specific parameter \textsc{CalTargetOffset} given in
Table~\ref{table:TAcaltargetoffset}.

The \tacq{PEAKXD} target exposure is then taken and the slew required to center the target is calculated and performed.
For NUV TAs, the FSW always uses the XD separation between the WCA and the target aperture for \texttt{STRIPE=B}. As a result,
NUV \tacq{PEAKXD} exposures taken with stripes other than \texttt{STRIPE=B} will be mis-centered in the XD direction
by as much as 2.5p (0.06\arcsec).
In addition, due to the slope of NUV spectra on the detector, targets with different spectral energy distributions will
center slightly differently ($\leq 2p = 0.05$\arcsec).
In the COS FSW, a single NUV parameter is used for the XD plate scale
for all gratings (\textsc{pcta\_NUVMilliArcsecsPerPixelXDisp}).
This value of this parameter was set in SMOV to 0.02384 mas/p.
The actual NUV plate scale is, however, grating dependent.
The estimates in Table~\ref{table:TAnuvPS} were derived in COS 2010-08 (Ghavamian, 2010).
The maximum difference from the FSW value is the G285M value, with a difference of $24.4-23.84 = 0.56$ mas/p.
This is a XD displacement of 2p (0.05\arcsec) in XD for a 1.1\arcsec\ G285M/\tacq{PEAKXD} adjustment.
For the FUV channel, the differences in XD plate scale are much larger and each grating
uses its own grating specific value of (\textsc{pcta\_FUVMilliArcsecsPerPixelXDisp}) as given in Table~\ref{table:TAscalar}.

There is one additional complication due to the off-axis alignment of the COS NUV channel.
The act of centering an NUV target in the XD direction moves the target in the AD direction
as given by the $\alpha_{yx}$ column in Table~\ref{table:TAnuvPS}. This parameter has units
of p/\arcsec\ of motion. In the worst case (G225M), a 1.25\arcsec\ PEAKXD centering will move the target 8.4p (0.20\arcsec)
in the AD. For this reason, NUV \tacq{PEAKXD}s should always precede \tacq{PEAKD}s. Also given in
Table~\ref{table:TAnuvPS} is the amount of motion in the XD direction for NUV AD motion
($\alpha_{xy}$). A 1.25\arcsec\ NUV \tacq{PEAKD} TA will move the target in the XD by a much smaller amount,
at most 0.4p (0.01\arcsec) for G185M.

Because of cross-contamination between the spectral stripes,
NUV \tacq{PEAKXD}s will not work properly for extended sources, and should be avoided.\clearpage

FUV \tacq{PEAKXD}s are performed in ``raw'' coordinates and do not receive the thermal and geometric corrections of
science data. The raw coordinate systems of each FUV segment are different.
\tacq{PEAKXD}s that use both segments must map the Segment-B detector ``Y'' (XD)
positions to Segment-A coordinates before determining the
combined Segment-A plus Segment-B WCS and PSA/BOA XD locations.
The FUV mapping is determined by the grating specific parameters \textsc{pcta\_XDispInterceptCoeff}
and \textsc{pcta\_XDispSlopeCoeff} (Table~\ref{table:TAscalar}).
Unlike the NUV, the FUV XD location determinations
give better results when using a mean to determine the XD locations (\textsc{pcta\_UseMedian4CAL4FUV=FALSE},
and \textsc{pcta\_UseMedian4PKXD4FUV=FALSE}). Because of the thermal and geometric distortions present
in raw FUV data, and the slight slope of FUV spectra on the detector, targets with different spectral
energy distributions will center slightly differently during FUV \tacq{PEAKXD}s. This effect has been
measured as {$\leq2~DE = 0.2$\arcsec}, less than our centering goal of $\pm0.3$\arcsec.

For the NUV, TA extraction subarrays isolate the \texttt{STRIPE} in use.
Table~\ref{table:TAnuvPEAKXDxc} gives the lower-left corner in $X_{detector}$ (XC) for all three stripes
for each grating and the WCA and PSA/BOA. All NUV PEAKXD subarrays have YC=0, YS=1024, and XS=81.

FUV TA subarrays must also exclude the geocoronal emission lines for each \texttt{SEGMENT}.
The subarrays described in the \tacq{SEARCH} section (\S~\ref{sec:SEARCH}) and Appendix~\ref{sec:TAsubs} are also used for \tacq{PEAKXD}.
The geocoronal lines excluded by the FUV TA subarrays are given in Table~\ref{table:GEO}.
FUV TA subarrays exclude only those lines above the double line in Table~\ref{table:GEO} (Daytime $> 0.2$~kR).
The observer should be aware of possible spectral contamination from the other lines, but these lines should not affect TA.


Since COS FUV TAs are done in raw coordinates (no thermal or geometric corrections), the
temperature at the time of the TA affects the XD location of the WCA and PSA/BOA spectra on
the detector. As outlined in COS-2010-13 (COS FUV and NUV On-Orbit Structural and Thermal Stability),
these digital locations will experience virtual oscillations of up to $\pm$ 0.05\arcsec\ in one orbit due to internal COS thermal instability.
The XD components of the four FUV stim locations move approximately in step with orbital phase, but with varying
magnitude As a result, the relative offset between the WCA and science aperture does not change as much as the absolute positions.
The overall XD centering error in \tacq{PEAKXD} due to thermal effects should be $\le 0.25$~DE (digital element), or $\le 0.025$\arcsec.
On the other hand, geometric distortions can affect overall XD centering by as much as 2~DE, or $0.2$\arcsec. Statistics on the
effects of geometric distortion on FUV XD \tacq{PEAKXD} centering are difficult to determine since the XD profiles are returned as a TA data product.

COS FUV on-orbit spectra are not perfectly aligned with the detector; there is a slight slope to both the WCA and PSA/BOA spectra ($\approx -0.27$ XD(DE) per 1000 AD(DE)).
This slope equates to an $\approx 3.5$ DE XD offset from the beginning to end of each segment.
This obviously broadens the XD profiles of both the WCA and PSA/BOA spectra, making centroiding
less precise and a function of the target spectral energy distribution (SED).
Since COS TA spectra are not geometrically corrected either, the XD profiles are further broadened
in non-linear fashion. FUV \tacq{PEAKXD} centering accuracy is therefore mildly \texttt{CENWAVE} dependent, even for the same target.

On-orbit, we have observed the centerings listed in Table~\ref{table:XDonorbit} with \tacq{PEAKXD} between days 2009.215--\reallastdate.
In all cases, the \tacq{PEAKXD} is following a spectroscopic \tacq{SEARCH}.
Overall, \tacq{PEAKXD} is correcting for \tacq{SEARCH} errors by XX$\pm$YY\arcsec,
clearly indicating its importance in properly centering targets in the XD.
The $1\sigma$ standard deviation indicates that \tacq{SEARCH} is centering the
target in the XD direction to within $\pm$0.3\arcsec\ about $90$\% of the time, but to within our desired
accuracy of $\pm$0.1\arcsec only $50$\% of the time.

By examining COS spectra, we can directly measure the final XD accuracies achieved by \tacq{PEAKXD} compared to the commanded XD positions.
The results for the NUV observations are given in Table~\ref{table:XDonorbitACCNUV} and the FUV results are given in Table~\ref{table:XDonorbitACCFUV}.
These results include all observations taken between 2009.215 and \reallastdate.
Due to incorrect WCA-to-PSA parameters, NUV BOA spectroscopic TAs will not correctly center the target in the aperture.
No BOA spectroscopic TAs are scheduled for Cycle~17. A PR will be filed to correct this before Cycle~18.
\begin{deluxetable}{|r|r|r|r|r|}
\tabcolsep 8pt
\tablecolumns{5}
\tablewidth{0pt}
\tablecaption{Average On-Orbit \tacq{PEAKXD} Centering Moves.\label{table:XDonorbit}}
\tabletypesize{\footnotesize}
\tablehead{
\colhead{Grating/Channel} & \colhead{Aperture} & \colhead{$|$XD Offset$|$} & \colhead{$\sigma_{|XD\ Offset|}$} & \colhead{\# of Centerings}
}
\startdata
\hline
\hline
G130M & PSA & -0.119\arcsec\ & 0.168\arcsec\ & 98 \\
G160M & PSA & -0.224\arcsec\ & 0.091\arcsec\ & 30 \\
G140L & PSA & -0.188\arcsec\ & 0.094\arcsec\ & 18 \\
\hline
G130M & BOA & -0.208\arcsec\ & 0.177\arcsec\ & 4 \\
G160M & BOA & \nodata\ & \nodata\ & 0 \\
G140L & BOA & \nodata\ & \nodata\ & 0 \\
\hline
FUV & PSA & -0.149" & 0.153" & 146 \\
FUV & BOA & -0.208" & 0.177" & 4 \\
\hline
\hline
PSA & ALL & -0.085\arcsec\ & 0.157\arcsec\ & 223\\
BOA & ALL & -0.179\arcsec\ & 0.166\arcsec\ & 5\\
\hline
\hline
%\hline
%G130M & PSA & -0.111\arcsec\ & 0.170\arcsec\ & 65 \\
%G160M & PSA & -0.207\arcsec\ & 0.072\arcsec\ & 22 \\
%G140L & PSA & -0.156\arcsec\ & 0.054\arcsec\ & 11 \\
%\hline
%G130M & BOA & -0.208\arcsec\ & 0.177\arcsec\ & 4 \\
%G160M & BOA &\nodata\ & \nodata\ & 0 \\
%G140L & BOA & \nodata\ & \nodata\ & 0 \\
%\hline
%FUV & PSA & -0.137\arcsec\ & 0.149\arcsec\ & 98 \\
%FUV & BOA & -0.208\arcsec\ & 0.177\arcsec\ & 4 \\
%\hline
%\hline
%PSA & ALL & -0.075\arcsec\ & 0.153\arcsec\ & 150\\
%BOA & ALL & -0.179\arcsec\ & 0.166\arcsec\ & 5\\
%\hline
%\hline
% Updated May 5, 2010
\enddata
\tablecomments{Previous NUV+FUV PSA+BOA TA stages centered the target to within $\pm$0.3\arcsec\
prior to \tacq{PEAKXD} 90\% of the time, and within $\pm$0.1\arcsec\ only 50\% of the time.}
\end{deluxetable}


\begin{deluxetable}{|r|r|r|r|r|r|r|r|}
\tabcolsep 4pt
\tablecolumns{8}
\tablewidth{0pt}
\tablecaption{FUV/PSA On-Orbit \tacq{PEAKXD} Accuracies.\label{table:XDonorbitACCFUV}}
\tabletypesize{\scriptsize}
\tablehead{
\colhead{FUV} & \colhead{Cenwave} & \colhead{$\Delta$XD$_{measured}$} & \colhead{$\sigma_{measured}$} &
\colhead{\# of ACQ/} & \colhead{$\Delta$XD$_{commanded}$} & \colhead{$\delta\Delta$XD} &\colhead{FUV}\\
\colhead{Grating} & \colhead{(\AA)} & \colhead{(PSA-WCA)} & \colhead{(PSA-WCA)} &
\colhead{PEAKXDs\tablenotemark{b}} & \colhead{(PSA-WCA)} & \colhead{(measured-commanded)} &\colhead{Segment}
}
\startdata
G130M &	1291 & -88.13 & 1.05 & 47 & -87.80 & 0.33 &FUVA \\
G130M &	1291 & -88.29 & 1.08 & 40 & -87.80 & 0.49 &FUVB \\
G130M &	1291 & -86.28 & 4.72 & 46 & -87.80 & -1.52 &FUVAB \\
\hline
G130M &	1300 & -88.29 & 1.07 & 23 & -87.80 & 0.49 &FUVA \\
G130M &	1300 & -88.55 & 1.38 & 23 & -87.80 & 0.75 &FUVB \\
G130M &	1300 & -88.08 & 1.25 & 23 & -87.80 & 0.28 &FUVAB \\
\hline
G130M &	1309 & -88.29 & 2.85 & 45 & -87.80 & 0.49 &FUVA \\
G130M &	1309 & -88.67 & 3.88 & 41 & -87.80 & 0.87 &FUVB \\
G130M &	1309 & -87.93 & 3.58 & 41 & -87.80 & 0.13 &FUVAB \\
\hline
G130M &	1318 & -87.31 & 1.14 & 15 & -87.80 & -0.49 &FUVA \\
G130M &	1318 & -88.45 & 1.32 & 15 & -87.80 & 0.65 &FUVB \\
G130M &	1318 & -87.60 & 1.18 & 15 & -87.80 & -0.20 &FUVAB \\
\hline
G130M &	1327 & -88.43 & 1.01 & 9 & -87.80 & 0.63 &FUVA \\
G130M &	1327 & -89.10 & 1.13 & 7 & -87.80 & 1.30 &FUVB \\
G130M &	1327 & -88.70 & 1.00 & 8 & -87.80 & 0.90 &FUVAB \\
\hline
G130M &	ALL & -88.01 & 1.28 & 135 & -87.80 & 0.21 &FUVA \\
G130M &	ALL & -88.43 & 1.57 & 124 & -87.80 & 0.63 &FUVB \\
G130M &	ALL & -87.79 & 1.44 & 124 & -87.80 & -0.01 &FUVAB \\

\hline
\hline
G160M &	1577 & -88.28 & 1.07 & 5 & -87.80 & 0.48 &FUVA \\
G160M &	1577 & -89.28 & 1.54 & 6 & -87.80 & 1.48 &FUVB \\
G160M &	1577 & -89.06 & 1.43 & 6 & -87.80 & 1.26 &FUVAB \\
\hline
G160M &	1589 & -88.21 & 2.38 & 41 & -87.80 & 0.41 &FUVA \\
G160M &	1589 & -89.29 & 2.71 & 45 & -87.80 & 1.49 &FUVB \\
G160M &	1589 & -89.77 & 3.02 & 49 & -87.80 & 1.97 &FUVAB \\
\hline
G160M &	1600 & -87.66 & 1.16 & 48 & -87.80 & -0.14 &FUVA \\
G160M &	1600 & -89.09 & 1.28 & 52 & -87.80 & 1.29 &FUVB \\
G160M &	1600 & -88.97 & 1.39 & 52 & -87.80 & 1.17 &FUVAB \\
\hline
G160M &	1611 & -88.99 & 1.07 & 26 & -87.80 & 1.19 &FUVA \\
G160M &	1611 & -89.35 & 1.20 & 24 & -87.80 & 1.55 &FUVB \\
G160M &	1611 & -89.42 & 1.31 & 24 & -87.80 & 1.62 &FUVAB \\
\hline
G160M &	1623 & -88.86 & 1.14 & 45 & -87.80 & 1.06 &FUVA \\
G160M &	1623 & -90.00 & 1.29 & 43 & -87.80 & 2.20 &FUVB \\
G160M &	1623 & -89.84 & 1.31 & 45 & -87.80 & 2.04 &FUVAB \\
\hline
G160M &	ALL & -88.34 & 1.33 & 163 & -87.80 & 0.54 &FUVA \\
G160M &	ALL & -89.37 & 1.41 & 167 & -87.80 & 1.57 &FUVB \\
G160M &	ALL & -89.42 & 1.46 & 171 & -87.80 & 1.62 &FUVAB \\
\hline
\hline
G140L &	1105 & -86.61 & 1.41 & 8 & -86.40 & 0.21 &FUVA \\
G140L &	1230 & -86.89 & 3.11 & 46 & -86.40 & 0.49 &FUVA \\
\hline
G140L &	ALL & -86.85 & 2.92 & 54 & -86.40 & 0.45 &FUVA \\
\hline
\hline
\enddata
\vspace{-0.5cm}
\tablecomments{Offsets are broken down by segment; FUVA, FUVB, or both (FUVAB).
Unless specified, \tacq{PEAKXD} uses both; unless the grating is G140L (Segment-A only).
Offsets and errors are in units of raw coordinate FUV Segment-A XD digital elements (DEs), which are $\sim$0.1\arcsec.
These results include all observations taken between 2009.215 and \reallastdate.}
\tablenotetext{a}{Only Segment-A is used for G140L TAs.}
\tablenotetext{b}{Only XD profiles with a signal-to-noise ratio greater than 40 are considered.}
\end{deluxetable}

COS observations begin to lose flux at an XD offset of $\sim$0.4\arcsec\ due to aperture vignetting.
As shown in Table~\ref{table:XDonorbitACCFUV}, the G160M \tacq{PEAKXD} currently have a systematic offset of 2-3 XD DE (0.2--0.3\arcsec).
This is most likely due to errors in the Segment B-to-A mapping parameters coupled with geometric distortion issues.
Uncertainties in the plate scale would produce a centered XD distribution with a large scatter.
An incorrect WCA-to-PSA offset would center the target incorrectly, but at the commanded position, not offset in a consistent direction.

In all cases, the \tacq{PEAKD}s in Figure~\ref{TAsmovPEAKD}\, followed spectroscopic \tacq{SEARCH}
and \tacq{PEAKXD}s.
\tacq{PEAKD} is correcting for \tacq{SEARCH} + \tacq{PEAKXD} AD residual centering errors by an average of \avgPEAKD.
Table~\ref{table:ADonorbit} gives the average and 1$\sigma$ standard deviations
of all \tacq{PEAKD} observations taken in the date range previously given, excluding
all \texttt{CENTER=RTB \tacq{PEAKD}s}.
Results are given for all COS gratings, apertures, and channels.
Some low signal-to-noise and high background (SAA) observations were removed in this analysis.
For normal observations taken with the PSA, the NUV \tacq{PEAKD} AD centerings averaged \avgPEAKDpsaNUV\ and for the FUV, the average AD centering was \avgPEAKDpsaFUV.
The $1\sigma$ standard deviations on all SMOV + GTO + GO observations tested
indicate after the initial spectroscopic \tacq{SEARCH}s, NUV+PSA observations were already
centered to the strictest NUV AD requirement of $<$ 0.041\arcsec\ 41\% of the time.
For the looser FUV requirement of $< $0.106\arcsec, COS TAs are already centered to the
required AD accuracy 82\% of the time.

\begin{deluxetable}{|r|r|r|r|r|}
\tabcolsep 8pt
\tablecolumns{5}
\tablewidth{0pt}
\tablecaption{Average On-Orbit \tacq{PEAKD} Centering Moves.\label{table:ADonorbit}}
\tabletypesize{\scriptsize}
\tablehead{
\colhead{Grating or Channel} & \colhead{Aperture} & \colhead{$|$AD Offset$|$} & \colhead{$\sigma_{|AD\ Offset|}$} & \colhead{\# of Centerings}
}
\startdata
G185M & PSA & 0.001\arcsec\ &0.156\arcsec\ &16\\
G225M & PSA & 0.060\arcsec\ &0.037\arcsec\ &9\\
G285M & PSA & -0.055\arcsec\ &0.000\arcsec\ &1\\
G230L & PSA & 0.011\arcsec\ &0.054\arcsec\ &15\\
\hline
G185M & BOA & 0.013\arcsec\ &\nodata\ &1\\
G225M & BOA & -0.002\arcsec\ &\nodata\ &1\\
G285M & BOA & -0.030\arcsec\ &0.072\arcsec\ &3\\
G230L & BOA & \nodata\ &\nodata\ &0\\
\hline
NUV & PSA & 0.016\arcsec\ &0.105 \arcsec\ &41\\
NUV & BOA & -0.016\arcsec\ &0.054 \arcsec\ &5\\
\hline
\hline
G130M & PSA & 0.007\arcsec\ &0.238 \arcsec\ &46\\
G160M & PSA & 0.017\arcsec\ &0.064 \arcsec\ &17\\
G140L & PSA & 0.053\arcsec\ &0.069 \arcsec\ &9\\
 \hline
G130M & BOA & 0.081\arcsec\ &0.119\arcsec\ &3\\
G160M & BOA & \nodata\ &\nodata\ &0\\
G140L & BOA & \nodata\ &\nodata\ &0\\
\hline
FUV & PSA & 0.015\arcsec\ &0.194 \arcsec\ &72\\
FUV & BOA & 0.081\arcsec\ &0.119 \arcsec\ &3\\
\hline
\hline
ALL & PSA &0.015\arcsec\ & 0.167\arcsec\ & 113\\
ALL & BOA &0.021\arcsec\ & 0.091\arcsec\ & 8\\
ALL & ALL &0.016\arcsec\ & 0.163\arcsec\ & 121\\
\hline
\hline
\enddata

\tablecomments{Only \tacq{PEAKD}s with \texttt{SCAN-SIZE}=5 and not \texttt{CENTER=RTB},
or \texttt{SCAN-SIZE}=3 and \texttt{CENTER=FW} taken between 2009.215--\reallastdate\ are included.
After the initial spectroscopic \tacq{SEARCH}s, NUV+PSA were already centered to the strictest NUV requirement of $<$ 0.041\arcsec\ 41\% of the time.
For the looser FUV requirement of $<$ 0.106\arcsec\ observations were already centered to the required accuracy 79\% of the time.}
\tablecomments{Some lower signal-to-noise and high background (SAA) observations were removed from the full sample to create this table.}
\end{deluxetable}

To test the on-orbit performance of \tacq{PEAKD}, a subsample of the aligned dwell points
shown in the bottom panels of Figure~\ref{TAsmovPEAKD} was created.
The NUV and FUV PSA profiles appear to be similar enough to merge the results of all \tacq{PEAKD}
trials together. BOA, \texttt{CENTER=RTB}, low signal-to-noise, SAA impacted \tacq{PEAKD}s,
and \tacq{PEAKD}s with \texttt{SCAN-SIZE=3} were removed from the sample.
All \texttt{CENTER=FWF} observations in the sample were converted back to detected counts on the sky
by adding the subtracted background (floor) number of counts back into the dwell counts.
The merged transmission versus AD profile is show in Figure~\ref{TAsmovPEAKDmerged}.

This AD profile was used to evaluate the performance of trial \tacq{PEAKD}s.
\texttt{SCAN-SIZE}s of 0.2--2.0 and \texttt{STEP-SIZE}s of 3, 5, 7, and 9 were
tested with all three \texttt{CENTER} options (\texttt{FW}, \texttt{FWF}, \texttt{RTB}).
The input distribution was taken to be that reported in Table~\ref{table:ADonorbit} (1$\sigma=$\avgPEAKDonesig).
A total of 30,000 normally distributed initial AD offset values were evaluated with simulated \tacq{PEAKD}s.
The results of these simulations are shown in Figure~\ref{TAsmovPEAKDperf}. The different
 \texttt{SCAN-SIZE}s are shown in different colors, and the different \texttt{CENTER}s are in
different panels. The abscissa is the \texttt{SCAN-SIZE}s, and the ordinate is the standard
deviation of the simulated centering error. Horizontal lines indicate our strictest FUV (0.106\arcsec)
and NUV (0.041\arcsec) centering requirements.

\clearpage
\vspace{-0.3cm}
\section{COS TA, Detector Backgrounds, and the SAA}\label{sec:TAback}
As outlined in detail in the ISRs COS 2010-11 (``COS NUV Detector Rates During SMOV and Early Cycle 17'') and
COS 2010-12 (``COS NUV Detector Dark Rates During SMOV and Early Cycle 17''),
the NUV and FUV detector backgrounds increase with proximity to the SAA, and the NUV count rate is slowly increasing with time.
The detector background has no effect on normal FUV target acquisitions.
NUV MIRRORB imaging TAs could be affected as outlined below.

%\subsubsection{COS Detector Background Evolution}\label{sec:newsubs}
Figure~\ref{FUVdarkcps} shows the FUV dark count rate measured inside the G130M/1309\AA\ FUV TA subarrays for
Segment~A ({\color{red} red} upward triangles), Segment~B ({\color{blue} blue} downward triangles), and
both ({\color{green} green} diamonds). Excepting the FUV dark count measurements taken between 2009.5--2009.7
(which included intentional excursions near the SAA during SMOV), the FUV dark count rate is constant for both
segments at around 3 counts/second over the TA subarrays.
Gaussian error bars ($\sqrt{counts}$) are included in the figure, but are smaller than the plotting symbols.\\

\clearpage

Discuss updated parameter table.
\clearpage
\section{Summary}\label{sec:Summary}
\vspace{-0.3cm}
In 2010, a series of minor adjustments are planned to further enhance the performance of COS TA based upon the performance of
GTO and GO TAs under on-orbit usage.
\begin{deluxetable}{|c|c|c|c|c|c|c|}
%\rotate
\tabcolsep 3pt
\tablecolumns{7}
\tablewidth{0pt}
\tablecaption{Recommended TA Options\label{table:TAoptions}}
\tabletypesize{\scriptsize}
\tablehead{
\colhead{Acquisition Type} &
\colhead{Description} & \colhead{SCAN-SIZE} & \colhead{STEP-SIZE (\arcsec)}&
\colhead{Parameters} &
\colhead{Recommendation} &
\colhead{Required S/N}
}
\startdata
ACQ/SEARCH & Spiral pattern; & 2 & 1.767 & \texttt{CENTER=FW,FWF,RTB} & FW & 40\\
&Multiple&3&1.767&&FWF& 40\\
&exposures&4&1.767&&FWF & 40\\
&&5&1.767&&FWF & 40\\
ACQ/PEAKXD & One exposure &N/A&N/A&FUV: \texttt{SEGMENT=A,B,BOTH} &BOTH& 40\\
&&N/A&N/A&NUV: \texttt{STRIPE=A,B,C} & B (MED)&40 to 100\\
ACQ/PEAKD & Linear pattern; & 3 & 1.5\tablenotemark{a} & \texttt{CENTER=FW,FWF,RTB} & FW & 40 to 100\\
&Multiple&5 & 1.1\tablenotemark{a} && FWF & \\
&exposures&7 & 1.0 && FWF&\\
&&9 & 0.7\tablenotemark{b} && FWF&\\
ACQ/IMAGE & Initial and &1&N/A&Aperture=PSA,BOA&& 40 (PSA)\\
   & confirm images &1&N/A&Mirror=A,B&& 40 60 (BOA)
\enddata
\tablenotetext{a}{The previously recommended value was 1.2\arcsec.}
\tablenotetext{b}{The previously recommended value was 1.0\arcsec.}
\end{deluxetable}

 Our updated recommendations are:
\begin{enumerate}
\item{All TA modes are providing good centering, although there is some room for improvement in certain modes.\footnote{See \#8 and 10, which discuss known problems with FUV/G160M and NUV/BOA spectroscopic TAs.}
If you need maximum wavelength accuracy, use NUV imaging mode,
otherwise use the mode that is fastest based upon STScI ETC simulation.}

\item{A single \tacq{SEARCH} is not sufficient to center a COS point-source target in the aperture; always follow up the first \tacq{SEARCH} with an \tacq{IMAGE}, \tacq{PEAKXD}+\tacq{PEAKD}, or a second 2x2 \tacq{SEARCH}.}

\item{With the correction to the FGS-to-COS offset (the SIAF update), COS \tacq{SEARCH} requirements can be relaxed to allow faster target centering. The recommended \tacq{SEARCH} \texttt{SCAN-SIZE} parameters listed in Table~\ref{table:ssgoal}, are based primarily upon the confidence observers have in their target coordinates.
Spending extra time to validate target coordinates is the best way save TA time.}

\item{Use \texttt{STRIPE}=B if at all possible for NUV spectroscopic \tacq{PEAKXD}s.}
\item{Because of potential cross-contamination between stripes, NUV spectroscopic TAs should not be used with extended sources.}
\item{Use \texttt{SCAN-SIZE}=5, \texttt{STEP-SIZE}=0.9, and \texttt{CENTER}=FWF for most \tacq{PEAKD} centerings.
For the most accurate AD centering possible, use \texttt{SCANSIZE}=9, \texttt{STEP-SIZE}=0.6, and \texttt{CENTER}=FWF.
Where minimal TA time is required, use \texttt{SCAN-SIZE}=3, \texttt{STEP-SIZE}=1.3, and \texttt{CENTER}=FW.}
\item{Signal-to-noise (S/N) is important to an accurate TA; use S/N=40 for PSA TAs, and S/N=60 for BOA.}
\item{FUV spectroscopic TAs that require absolute photometric accuracy should avoid G160M \tacq{PEAKXD}s as these can occasionally be partially vignetted.
The \tacq{PEAKXD} phase of FUV TA may not be able to properly center the target in the XD direction.
This is partially due to errors in the Segment B-to-A mapping parameters, which should be updated.}
\item{The WCA+MIRRORB \texttt{LTAIMCAL} TA subarrays should be reduced to avoid contamination from the rising NUV detector background. The \texttt{LTAIMAGE} subarrays could also be reduced; this would not affect the quality of the TA, but would reduce processing time of the centering step.\label{item:LTAIMCAL}}
\item{The NUV imaging \tacq{SEARCH} TA subarrays should be reduced to avoid contamination from the rising NUV detector background.}

\item{Due to incorrect WCA-to-PSA parameters, NUV BOA spectroscopic TAs will currently not correctly center the target in the aperture. No BOA spectroscopic TAs are scheduled for Cycle~17, and this will be corrected before Cycle~18.}
\item{The current COS SAA contours should be adjusted to prevent any COS TAs from occuring in the elevated background region that currently exists over South America. COS TAs that use \tacq{SEARCH} and/or \tacq{PEAKD} will not center the target correctly in this region.\footnote{COS SAA contours were adjusted in PRs\# 65147, 65227, and 65188 on May 13--25, 2010}}
\item{The number of counts in the primary spot from the wavelength calibration
lamp in \texttt{LTAIMCAL} in \texttt{MIRRORB} mode has been decreasing since SMOV. Reducing the size of the \texttt{LTAIMCAL} subarrays is planned (\ref{item:LTAIMCAL}, above) and will help with signal-to-noise issues, but may not be enough to negatively impact \tacq{IMAGE} + \texttt{MIRRORB} centering (one of the most used TA modes). The number of counts determined to have come from the wavelength calibration lamp during \texttt{LTAIMCAL} should be monitored, and the exposure time for \texttt{LTAIMCAL}+ \texttt{MIRRORB} may need to be increased for Cycle~19 and beyond.}
\normalsize
\end{enumerate}

\clearpage
\section{Change History for \thisISR}
Version 0.5: April 2012 - Original Document
Version 0.6: May 2018
\clearpage
\section{References}

\vspace{-0.23cm} \noindent Keyes, C. \& Penton, S. 2010, COS ISR 2010-14(v1), 
\clearpage
\appendix
\vspace{-0.3cm}
\section{Appendix A: COS TA Subarrays}\label{sec:TAsubs}
\vspace{-0.3cm}
%\input{TA2_subarrays.tex}
\clearpage
\vspace{-0.4cm}
\section{Appendix B: Description of TA FSW Parameters}\label{sec:FSWdescription}
\vspace{-0.5cm}
In this Appendix, the flight software (FSW) parameters are described.
In Appendix~\ref{sec:FSWparameters}, the current and historic values of the parameters are given. A description of each parameter is followed by its \textbf{Format}, \textbf{Units}, \textbf{Limits/Ranges}, \textbf{Scaling}, and optionally usage \textbf{Note}s. The majority of these sections are taken directly from the FSW code. Section~\ref{sec:Vector} describes the FSW vector parameters, and \S\ref{sec:Scalar} lists the scalar parameters. In some cases, parameters that were originally scalar parameters in the FSW were replaced by vector (usually grating or central wavelength specific) parameters through a change in ground commanding. These parameters are noted as such in \S\ref{sec:Scalar}.
\vspace{-0.4cm}
%\input{ta_param.tex}
\clearpage
\vspace{-0.3cm}
\section{Appendix C: Values of TA FSW Parameters}\label{sec:FSWparameters}
\vspace{-0.3cm}
In this section, the historic and current values of the TA parameters described in Appendix~\ref{sec:FSWdescription} are presented. Where appropriate, the Problem Report (PR) number is given to help document the reasons for the parameter changes. In addition, the day of the year 2009 of the change is given as SMS\textit{day\_of\_year\_2009}, e.g., SMS194.
\vspace{-0.3cm}

\subsection{WCA-to-PSA/BOA Offsets}
The only vector parameter in the current FSW is an array which relates the relative XD offset between the WCA and the PSA or BOA. The offsets given in Table~\ref{table:TAcaltargetoffset} are grating, but not central-wavelength dependent. In the future, these parameters may become central wavelength dependent for the FUV channel. The offsets given in Table~\ref{table:TAcaltargetoffset} are in units of 0.1p, NUV, or XD digital elements (DE), FUV.

\begin{deluxetable}{|l|r|r|r|l|}
\tablewidth{0pt}
\tabcolsep 12pt
\tablecolumns{5}
\tabletypesize{\footnotesize}
\tablecaption{WCA-to-PSA/BOA Offsets.\label{table:TAcaltargetoffset}}
\tablehead{\colhead{Mechanism}&\colhead{Detector}& \colhead{PSA} & \colhead{BOA} &\colhead{History}}
\startdata
OSM1\_G130M & FUV & -833 & -833& TVAC 2006 \\
			&		& -822 &  & SMOV--SMS194 (PR\#63036)\\
			&		& -878 & -860& SMOV--SMS201 (PR\#63095)\\
OSM1\_G140L & FUV & -833 & -833& TVAC 2006 \\
			&		& -850 & -832& SMOV--SMS194\\
			&		& -864 & -846& SMOV--SMS201 (PR\#63095)\\
OSM1\_G160M & FUV & -833 & -833& TVAC 2006 \\
			&		& -816 & -798& SMOV--SMS194\\
			&		& -878 & -860& SMOV--SMS201 (PR\#63095)\\
OSM2\_G185M & NUV & 3712 & 3712& TVAC 2006 \\
			&		& 3720 &  & SMOV--SMS194 (PR\#63036)\\
			&		& 3742 & 3458& SMOV--SMS201 (PR\#63095)\\
			&		&		 &	 \textbf{3692}	 & PR\#65152 \\
OSM2\_G225M & NUV & 3712 & 3712& TVAC 2006 \\
			&		& 3677 & 3597& SMOV--SMS194\\
			&		& 3746 &  3462& SMOV--SMS201 (PR\#63095)\\
			&		&		 &	\textbf{3689} & PR\#65152\\
OSM2\_G230L & NUV & 3712 & 3712& TVAC 2006 \\
			&		& 3647 &  & SMOV--SMS194 (PR\#63036)\\
			&		& 3734 &  3450& SMOV--SMS201 (PR\#63095)\\
			&		&		 &		\textbf{3685} & PR\#65152\\
OSM2\_G285M & NUV & 3712 & 3712& TVAC 2006 \\
			&		& 3691 & 3611& SMOV--SMS194\\
			&		& 3749 & 3465& SMOV--SMS201 (PR\#63095)\\
			&		&		 &	 \textbf{3667}& PR\#65152\\
\enddata %UPdated April 14, 2010
\tablecomments{Table values are XD offsets (\textsc{pcta\_CalTargetOffset)} in units of 0.1p (NUV) or 0.1DE (FUV). The NUV BOA offsets where updated with PR\#65152, and will be implemented before Cycle~18. No BOA spectroscopic TAs are scheduled for Cycle~17.}
\end{deluxetable}
\clearpage
\end{document}
